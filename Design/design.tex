\chapter{Design}

\section{Overall System Design}

\begin {itemize}
	\item Main Parts
	\begin {itemize}
		\item Database View Interface
		\begin {itemize}
			\item Presented as a database window with 6 push buttons that open dialog boxes for different functions
			\item the 6 push buttons are labelled:
			\begin {itemize}
				\item Open
				\item Add
				\item Edit
				\item delete
				\item Search
				\item Print
			\end {itemize}
		\item Each table in the database is displayed in a different tab
		\item Main window displays database in a box above function buttons
		\item Has a menu bar at the top that provides shortcuts to the functions and an about section with a help section that contains a user manual
	\item Open Interface
		\item Two push buttons - open file and close
		\item 1st button opens the file browser to chose a database to open
		\item 2nd button closes the dialog button
	\item Add Interface
		\item Drop down menu to select table
		\item Fields with editable text boxes/line edits to enter attributes
		\item confirm button
	\item Edit Box
				\item Drop down menu to select table
		\item Fields with editable text boxes/line edits to enter attributes
		\item confirm button
	\item Delete Box
		\item Select table drop down menu
		\item Select item drop down menu
		\item Delete button
		\item Delete all itmes button
		\item require a password to b entered first in a dialog box
	\item Search
		\item Select table drop down menu
		\item Search term QLineEdit
		\item Search push button
		
		\end {itemize}
	\end {itemize}
\end {itemize}

\subsection{Short description of the main parts of the system}






\subsection{System flowcharts showing an overview of the complete system}

\section{User Interface Designs}

\section{Hardware Specification}

\section{Program Structure}

\subsection{Top-down design structure charts}

\subsection{Algorithms in pseudo-code for each data transformation process}

\subsection{Object Diagrams}

\subsection{Class Definitions}

\section{Prototyping}

\section{Definition of Data Requirements}

\subsection{Identification of all data input items}

\subsection{Identification of all data output items}

\subsection{Explanation of how data output items are generated}

\subsection{Data Dictionary}

\subsection{Identification of appropriate storage media}

\section{Database Design}

\subsection{Normalisation}

\subsubsection{ER Diagrams}

\subsubsection{Entity Descriptions}

\subsubsection{1NF to 3NF}

\subsection{SQL Queries}

\section{Security and Integrity of the System and Data}

\subsection{Security and Integrity of Data}

\subsection{System Security}

\section{Validation}

\section{Testing}

\begin{landscape}
\subsection{Outline Plan}

\begin{center}
    \begin{tabular}{|p{2cm}|p{5cm}|p{5cm}|p{4cm}|}
        \hline
        \textbf{Test Series} & \textbf{Purpose of Test Series} & \textbf{Testing Strategy} & \textbf{Strategy Rationale}\\ \hline
        Example & Example & Example & Example \\ \hline
    \end{tabular}
\end{center}

\subsection{Detailed Plan}

\begin{center}
    \begin{longtable}{|p{1.5cm}|p{2.5cm}|p{2.5cm}|p{2cm}|p{2cm}|p{2cm}|p{2cm}|p{2cm}|}
        \hline
        \textbf{Test Series} & \textbf{Purpose of Test} & \textbf{Test Description} & \textbf{Test Data} & \textbf{Test Data Type (Normal/ Erroneous/ Boundary)} & \textbf{Expected Result} & \textbf{Actual Result} & \textbf{Evidence}\\ \hline
        Example & Example & Example & Example & Example & Example & Example & Example \\ \hline
    \end{longtable}
\end{center}
\end{landscape}