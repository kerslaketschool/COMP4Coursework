 \chapter{Design}

\section{Overall System Design}

\subsection{Short description of the main parts of the system}

\begin {itemize}
	\item Main Parts
	\begin {itemize}
		\item Database View Interface
		\begin {itemize}
		    \item The main window for accessing the entire system. It is designed so that all the dialog boxes and other windows can be opened from wihin it and gives a main, tabbed layout for the tables.
			\item Presented as a database window with 6 push buttons that open dialog boxes for different functions
			\item the 6 push buttons are labelled:
			\begin {itemize}
				\item Open
				\item Add
				\item Edit
				\item delete
				\item Search
				\item Print
			\end {itemize}
		\item Each table in the database is displayed in a different tab
		\item Main window displays database in a box above function buttons
		\item Has a menu bar at the top that provides shortcuts to the functions and an about section with a help section that contains   a user manual
	\item Add Interface
	\begin {itemize}
	    \item Dialog box that contains forms that allow the user to enter data specific to the table they want to add an item to 
		\item Drop down menu to select table
		\item Fields with editable text boxes/line edits to enter attributes
		\item confirm button
	\end {itemize}
	\item Edit Interface
    \begin {itemize}
	    \item Dialog box that contains forms that allow the user to edit data specific to the table they want to add an item 
		\item Drop down menu to select table
		\item Fields with editable text boxes/line edits to enter attributes
		\item confirm button
	\end {itemize}
	\item Delete Interface
	\begin {itemize}
	    \item Dialog box allowing the user to delete a specific item from any of the tables
		\item Select table drop down menu
		\item Select item drop down menu
		\item Delete button
		\item Delete all itmes button
		\item require a password to b entered first in a dialog box
	\end {itemize}
	\item Search Interface
	\begin {itemize}
	    \item Dialog box allowing the user to search through any or all of the tables to find a specific search term/query
		\item Select table drop down menu
		\item Search term QLineEdit
		\item Search push button
	\end {itemize}
	\item Print Interface
	\begin{itemize}
	    \item Dialog box allowing the user to select a type of form to print out based on selected information like a list of members, an invoice, or a regime.
	    \item 3 Combo boxes allowing the user to select the type of form, the table they want the information from, and the information.
	    \item A push button allowing the user to confirm they want to print the selected details.
	\end {itemize}
	\item Password Interface
	\begin {itemize}
	    \item Dialog box that presents the user an outlet for entering the password for the system
	    \item Enter password lineEdit allowing the user to enter their password
	    \item Enter password push Button allowing the user to confirm the password they've entered
	\end {itemize}
	\end {itemize}
	\end {itemize}
\end {itemize}

\subsection{System flowcharts showing an overview of the complete system}

\begin{figure}[H]
    \includegraphics[width=\textwidth]{flowchart 5.jpg}
    \caption{Flowchart} \label{fig:Flowchart}
\end{figure}


\begin{figure}[H]
    \includegraphics[width=\textwidth]{flowchart4.jpg}
    \caption{Flowchart} \label{fig:Flowchart}
\end{figure}

\begin{figure}[H]
    \includegraphics[width=\textwidth]{flowchart3.jpg}
    \caption{Flowchart} \label{fig:Flowchart}
\end{figure}

\begin{figure}[H]
    \includegraphics[width=\textwidth]{flowchart 2.jpg}
    \caption{Flowchart} \label{fig:Flowchart}
\end{figure}

\begin{figure}[H]
    \includegraphics[width=\textwidth]{flowchart1.jpg}
    \caption{Flowchart} \label{fig:Flowchart}
\end{figure}

\section{User Interface Designs}

\begin{figure}[H]
    \includegraphics[width=\textwidth]{Gui_1.JPG}
    \caption{GUI Design} \label{fig:GUI Designs}
\end{figure}

\begin{figure}[H]
    \includegraphics[width=\textwidth]{Gui_2.JPG}
    \caption{GUI Design} \label{fig:GUI Designs}
\end{figure}

\begin{figure}[H]
    \includegraphics[width=\textwidth]{Gui_3.JPG}
    \caption{GUI Design} \label{fig:GUI Designs}
\end{figure}

\begin{figure}[H]
    \includegraphics[width=\textwidth]{Gui_4.JPG}
    \caption{GUI Design} \label{fig:GUI Designs}
\end{figure}

\begin{figure}[H]
    \includegraphics[width=\textwidth]{Gui_5.JPG}
    \caption{GUI Design} \label{fig:GUI Designs}
\end{figure}

\begin{figure}[H]
    \includegraphics[width=\textwidth]{Gui_6.JPG}
    \caption{GUI Design} \label{fig:GUI Designs}
\end{figure}

\begin{figure}[H]
    \includegraphics[width=\textwidth]{Gui_7.JPG}
    \caption{GUI Design} \label{fig:GUI Designs}
\end{figure}

\begin{figure}[H]
    \includegraphics[width=\textwidth]{Gui_8.JPG}
    \caption{GUI Design} \label{fig:GUI Designs}
\end{figure}

\begin{figure}[H]
    \includegraphics[width=\textwidth]{Gui_9.JPG}
    \caption{GUI Design} \label{fig:GUI Designs}
\end{figure}

\section{Hardware Specification}

The owner currently uses a Notebook laptop powered by an intel 1st generation
i5 processor and 4GB of RAM. He already uses this laptop to run his current
system which uses more resources and power than this system will hopefully
require, but if not he still has more than enough horse power to run the new
system. His use of a laptop is convienient as it means his system is portable
so he doesn’t need to do any data entry (if required) confined to his office and
away from his client if he doesn’t want too and it also means he has a battery so
in the case of a power outtage he wont lose any data. Though it is worth noting
that he may soon be upgrading to a more powerful desktop soon so while this
lowers portability, he has more power for the proposed system and he can easily
have a battery backup in the form of a UPS(Uninterptable Power Supply).
All members will be running this program off the same workstation so the
databases will be stored locally. This wont be a problem as the laptop has a 500 gb hard drive that wont be filled up too quickly and has lots of room for databases and the program. The program will be ouput through a display with a resolution of at least 600 * 800 (like the one used in the laptop) to accomodate the amount of screen real estate the program will require and the program will be operated and have data inputed using mouse/trackpad and keyboard (also contained within my clients laptop). The system will not use any additional peripherals. The system will also need to run on a windows os (windows 7, 8, 8.1, or 10) as thats what its been developed on and intended to run on, though it may also be compatable with MacOS and many Linux distributions as python is fairly portable and compatable with all of those languages. Though its worth noting that the program will only be compiled to work on a windows pc as thats the operating system used by my client.



\section{Program Structure}

\subsection{Top-down design structure charts}

\begin{figure}[H]
    \includegraphics[width=\textwidth]{StructureChart1.jpg}
    \caption{Structure Chart} \label{fig:StructureChart}
\end{figure}

\begin{figure}[H]
    \includegraphics[width=\textwidth]{StructureChart2.jpg}
    \caption{Structure Chart} \label{fig:StructureChart}
\end{figure}



\subsection{Algorithms in pseudo-code for each data transformation process}

\begin{python}

START

Function searchDatabase(table, info, searchTerm, database)
    currentDatabase = database
    connect database as db
    cursor.execute <--- """select from '{0}' where '{1}' = ‘{2}’ """.format(table, info, SearchItem)
    results <-- cursor.fetchall

END

\end{python}


\subsection{Object Diagrams}

\begin{figure}[H]
    \includegraphics[width=\textwidth]{RelationshipDiagram1.jpg}
    \caption{Relationship Diagram} \label{fig: Relationship Diagram}
\end{figure}

\subsection{Class Definitions}

\begin{figure}[H]
    \includegraphics[width=\textwidth]{NewClassDefs.JPG}
    \caption{Class Definitions} \label{fig:Class Definitions}
\end{figure}

\section{Prototyping}

For the prototype I created various functional GUI's to test how my program will look, feel and operate for my client and serveral commandline python functions to test some of the functionality. This allowed me to make sure the program was coming along to my clients expectations.

\begin{figure}[H]
    \includegraphics[width=\textwidth]{PrototypeGui1.JPG}
    \caption{Prototype Main Gui Window} \label{fig:Prototype Main Gui Windows}
\end{figure}

This shows the window for my main Gui. The blank white space is where the database will go once the program is complete, with each tab representing a different table. My client used this and seemed pleased with the potential of the programm but felt like he needed some functionality to get an idea of how the program will work.

\begin{figure}[H]
    \includegraphics[width=\textwidth]{PrototypeGui2.JPG}
    \caption{Prototype Search Commandline Window} \label{fig:Prototype Search Commandline Window}
\end{figure}

This shows a commandline interface for a search function I made to show the potential of the functionality of the program for my client. He seemed impressed by the potential problems this search function could solve for when he needs to find any specific information quickly like how in this example he could find out everyone who had outstanding payments for membership.

\section{Definition of Data Requirements}

\subsection{Identification of all data input items}

\begin{itemize}
    \item Member Name
    \item Member Address
    \item Member Telephone Number
    \item Type of Membership
    \item Join Date
    \item Date of Induction
    \item How Paid
    \item Amount Paid
    \item Registration Fee
    \item Registration Date
    \item Payment Type
    \item Date of Payment
    \item How Much
    \item Paid?
    \item Exercise Name
    \item Exercise Description
    \item Specific Description
    \item Start Date
    \item End Date
\end{itemize}

\subsection{Identification of all data output items}

\begin{itemize}
    \item Member Details Printout
    \item Invoice Printout
    \item Memberlist Printout
    \item Member Regime
\end{itemize}

Output to Database

\begin{itemize}
    \item Member Name
    \item Member Address
    \item Member Telephone Number
    \item Type of Membership
    \item Join Date
    \item Date of Induction
    \item How Paid
    \item Amount Paid
    \item Registration Fee
    \item Registration Date
    \item Payment Type
    \item Date of Payment
    \item How Much
    \item Paid?
    \item Exercise Name
    \item Exercise Description
    \item Specific Description
    \item Start Date
    \item End Date

\subsection{Explanation of how data output items are generated}

\begin{figure}[H]
    \includegraphics[width=\textwidth]{Outputexplanations1.JPG}
    \caption{Output Explanations} \label{fig:Output Explanations}
\end{figure}

\begin{figure}[H]
    \includegraphics[width=\textwidth]{Outputexplanations2.JPG}
    \caption{Output Explanations} \label{fig:Output Explanations}
\end{figure}

\subsection{Data Dictionary}

\begin{figure}[H]
    \includegraphics[width=\textwidth]{datadict1.JPG}
    \caption{Data Dictionary} \label{fig: Data Destinations and Sources }
\end{figure}

\begin{figure}[H]
    \includegraphics[width=\textwidth]{NewDict2.JPG}
    \caption{Data Dictionary 2} \label{fig: Data Destinations and Sources 2 }
\end{figure}

\begin{figure}[H]
    \includegraphics[width=\textwidth]{NewDict3.JPG}
    \caption{Data Dictionary 3} \label{fig: Data Destinations and Sources 2 }
\end{figure}

\subsection{Identification of appropriate storage media}

The system only needs to be accessed by the one workstation in my clients gym meaning that the database files can be stored locally as he won't need to access the files on any other system. Although in saying that the files will be backed up in the gym owners dropbox account so that if the system the databases are stored on breaks then the most recent version of the files will be retrievable and then the files can be accessed on a different system in case of an emergency, adding a certain level of security. 

\section{Database Design}

\subsection{Normalisation}

\subsubsection{ER Diagrams}

\begin{figure}[H]
    \includegraphics[width=\textwidth]{RelationshipDiagram1.jpg}
    \caption{ER Diagram} \label{fig: ER Diagram}
\end{figure}

\subsubsection{Entity Descriptions}

Membership Details(\textbf{\underline{Membership No.}}, Last Name, First Name, Address, Telephone No., Type Of Membership, Date of Induction, When Joined, How Paid,amount, registration fee, Registration Date, Payment Type, Comments)

Payment Details(\textbf{\underline{Membership No.}},\textbf{\underline{Date of Payment}}, How Much, Paid)

Regime(\textbf{\underline{Membership No.}},\textbf{\underline{ExerciseID}},Specific Description, \textbf{\underline{Start Date}}, End Date)

Exercise(\textbf{\underline{ExerciseID}}, Name, Description)


\subsubsection{1NF to 3NF}

\begin{figure}[H]
    \includegraphics[width=\textwidth]{Norm1.JPG}
    \caption{Unnormalised} \label{fig: Unormalised}
\end{figure}

\begin{figure}[H]
    \includegraphics[width=\textwidth]{Norm2.JPG}
    \caption{1st Normal Form} \label{fig: 1st Normal Form}
\end{figure}

\begin{figure}[H]
    \includegraphics[width=\textwidth]{Norm3.JPG}
    \caption{2nd Normal Form} \label{fig: 2nd Normal Form}
\end{figure}

\begin{figure}[H]
    \includegraphics[width=\textwidth]{Norm4.JPG}
    \caption{3rd Normal Form} \label{fig: 3rd Normal Form}
\end{figure}

\subsection{SQL Queries}

\begin{figure}[H]
    \includegraphics[width=\textwidth]{Queries1.JPG}
    \caption{SQL Queries} \label{fig:SQL Queries}
\end{figure}

\begin{figure}[H]
    \includegraphics[width=\textwidth]{Queries2.JPG}
    \caption{SQL Queries} \label{fig:SQL Queries}
\end{figure}

\section{Security and Integrity of the System and Data}

\subsection{Security and Integrity of Data}
The system will have data entered through combo boxes/drop down menus where possible to avoid the user entering bad data but for a large amount of the fields (like a members name) has to be a raw input typed by the user with a keyboard. All the data input will be checked for errors. e.g making sure the data type is correct - strings can't be entered where an integer is required. 

\subsection{System Security}
The system will be protected behind a password that is only known by certain users/members of staff at the gym and a secondary password will be required to use certain functions like the function to delete any of the data or edit anything important.

\section{Validation}

\begin{figure}[H]
    \includegraphics[width=\textwidth]{Validation1.JPG}
    \caption{Validation} \label{fig:Valdation}
\end{figure}

\begin{figure}[H]
    \includegraphics[width=\textwidth]{Validation2.JPG}
    \caption{Validation} \label{fig:Valdation}
\end{figure}

\begin{figure}[H]
    \includegraphics[width=\textwidth]{Validation3.JPG}
    \caption{Validation} \label{fig:Valdation}
\end{figure}

\section{Testing}

\begin{landscape}
\subsection{Outline Plan}

\begin{center}
    \begin{tabular}{|p{2cm}|p{5cm}|p{5cm}|p{4cm}|}
        \hline
        \textbf{Test Series} & \textbf{Purpose of Test Series} & \textbf{Testing Strategy} & \textbf{Strategy Rationale}\\ \hline
        Example & Example & Example & Example \\ \hline
    \end{tabular}
\end{center}

\subsection{Detailed Plan}

\begin{center}
    \begin{longtable}{|p{1.5cm}|p{2.5cm}|p{2.5cm}|p{2cm}|p{2cm}|p{2cm}|p{2cm}|p{2cm}|}
        \hline
        \textbf{Test Series} & \textbf{Purpose of Test} & \textbf{Test Description} & \textbf{Test Data} & \textbf{Test Data Type (Normal/ Erroneous/ Boundary)} & \textbf{Expected Result} & \textbf{Actual Result} & \textbf{Evidence}\\ \hline
        Example & Example & Example & Example & Example & Example & Example & Example \\ \hline
    \end{longtable}
\end{center}
\end{landscape}